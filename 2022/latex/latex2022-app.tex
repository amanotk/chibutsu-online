%Appendix
%%%%%%%%%%%%%%%%%%%%%%%%%%%%%%%%%%%%%%%
%%%%%%%%%%%%%%%%%%%%%%%%%%%%%%%%%%%%%%%

\appendix


\section{文字コードの設定}
texファイルの文字コードが適切でないと(559のeduではUTF-8でないと)、文字化けを起こしてしまいます。最悪の場合、platexの段階でエラーを起こしてdviファイルが出来上がらないこともあります。ここではその対処法をまとめておきます。

\subsection{Emacsの文字コードの設定}
以下のコマンドは、打った文字をそのままに、保存する文字コードをUTF-8に設定するものです。
\begin{screen}
\keytop{Ctrl}$+$\keytop{x} \keytop{Enter} \keytop{f}
\end{screen}
しかし、この方法ではEmacsのウィンドウを閉じるたびに文字コードがEUC-JPに戻ってしまいます。毎回これを打ち直すのは面倒ですから、設定ファイル\verb+.emacs+を覗きにいきましょう。

Emacsで.emacsを開きましょう。この中から、
\begin{screen}
\begin{verbatim}
;; YaTeX-mode
(setq auto-mode-alist
      (cons (cons "\\.tex$" 'yatex-mode) auto-mode-alist))
(setq dvi2-command "xdvi"
      tex-command "platex"
      dviprint-command-format "dvips %s -o !lpr -Pionia"
      YaTeX-kanji-code 3)
\end{verbatim}
\end{screen}
を見つけてください。この中の、\underline{YaTeX-kanji-code 3}が、YaTeXでEUC-JPを使うように、と設定しています。ここを\underline{YaTeX-kanji-code 4}と書き換えます。
\begin{screen}
\begin{verbatim}
;; YaTeX-mode
(setq auto-mode-alist
      (cons (cons "\\.tex$" 'yatex-mode) auto-mode-alist))
(setq dvi2-command "xdvi"
      tex-command "platex"
      dviprint-command-format "dvips %s -o !lpr -Pionia"
      YaTeX-kanji-code 4)
\end{verbatim}
\end{screen}
これでYaTeXがUTF-8を使ってくれるようになります。

先ほどのコマンドとそっくりですが、
\begin{screen}
\keytop{Ctrl}$+$\keytop{x} \keytop{Enter} \keytop{r}
\end{screen}
というものがあります。これはEmacsでファイルを開いたら文字化けした場合に使います。「UTF-8で書いたはずのものが文字化けした」というときにはこれを使って、ミニバッファに`utf-8'と入力すれば良いです。何か聞かれたら`yes'と答えておきましょう。

興味のある人は、\verb+.emacs+内の、
\begin{screen}
\begin{verbatim}
(set-default-coding-systems    'utf-8)
(set-terminal-coding-system    'utf-8)
(set-keyboard-coding-system    'utf-8)
(set-buffer-file-coding-system 'utf-8)
(set-clipboard-coding-system   'utf-8)
(setq default-enable-multibyte-characters t)
\end{verbatim}
\end{screen}
がどういう意味なのか調べてみてください。

\subsection{ターミナルの文字コードの設定}
既に環境変数についてならっていることと思います。ターミナルの文字コードは、この環境変数で設定できます。

現在の文字コードを確認するには、ターミナルで\underline{echo \$LANG}とします。
'\verb+ja_JP.EUC-JP+'と表示されたらEUC-JP、`\verb+ja_JP.utf8+'と表示されたらUTF-8です。UTF-8にしたければ、ターミナル上で\underline{export LANG=ja\_JP.utf8}とします。しかし、この方法では、変更はターミナルを開いている間のみ有効で、一度ターミナルを閉じると、再びEUC-JPに設定されてしまいます。

さて、ホームディレクトリに\verb+.bashrc+というファイルがありましたね。この中身をのぞくと、下の方に
\begin{screen}
\begin{verbatim}
export LANG=ja_JP.EUC-JP
\end{verbatim}
\end{screen}
という行があると思います。ターミナルの文字コードは環境変数\verb+LANG+によって決められており、\verb+.bashrc+によって現在はEUC-JPに設定されています。そこでこの行の頭に\#をつけてコメントアウトし、直後に一行付け加えて、
\begin{screen}
\begin{verbatim}
#export LANG=ja_JP.EUC-JP
export LANG=ja_JP.utf8
\end{verbatim}
\end{screen}
とすることで、ターミナルを立ち上げるたびにUTF-8に設定されるようになります。今すぐにこの変更を適用するには、\underline{source .bashrc}が必要でしたね。

\pagebreak


\section{執筆支援環境}
これまで見てきたとおり、{\LaTeX}の文書はただ文を入力すればよいのではありません。
これは初学者にとっては非常に疲れることです。EmacsにはYaTeXと呼ばれる素晴らしい執筆支援環境があるので紹介します。

\subsection{EmacsとYaTeX(野鳥)}
興味のある人はホームディレクトリにある
.emacsというファイルを\underline{less .emacs}とかでのぞいてみてください。
そこにEmacsでの{\LaTeX}の設定などが書いてあります。
かなりマニアックですが…。

YaTeXの細かい機能は
\begin{screen}
\begin{verbatim}
http://www.yatex.org/
\end{verbatim}
\end{screen}
からたどれますが、この中から主な機能を抜き出すと次のようになります。
以下、例えば\keytop{SPC}はスペースキーを押すことをあらわします。

\subsubsection{補完入力}
\begin{description}
 \item[begin型補完]
   \keytop{Ctrl}$+$\keytop{C} \keytop{Ctrl}$+$\keytop{B} \keytop{SPC}と入力すると、ミニバッファで
   \begin{screen}
     {\tt Begin environment(default document):}
   \end{screen}
   とか聞いてきます。この状態でそのまま\keytop{Enter}を入力すれ
   ば、
   \begin{screen}
\begin{verbatim}
\begin{document}

\end{document}
\end{verbatim}
   \end{screen}
   のような\verb+\begin{環境名} +$\cdots$
   \verb+\end{環境名}+の組み合わせが出てきま
   す。もし、自分が入力したいのはほかの環境名だぜ、というときは、
   その環境の名前({\tt equation}とか)を入力して\keytop{Enter}
   すればよいのですが、たいていは頭文字({\tt equation}なら
   \keytop{E})と\keytop{Tab}で補完してくれます。

 \item[section型補完]
   \keytop{Ctrl}$+$\keytop{C} \keytop{Ctrl}$+$\keytop{S} と入力すると
   \begin{screen}
     {\tt (C-v for view-section) ???\{\}(default documentclass):}
   \end{screen}
   とか聞いてくるので、以下同じ。

 \item[large型補完]
   \keytop{Ctrl}$+$\keytop{C} \keytop{Ctrl}$+$\keytop{L}と入力すると,
   \begin{screen}
     {\tt \{??? \}(default large):}
   \end{screen}
   とか聞いてきます。もう予想はつくと思いますが、ここで
   \keytop{Enter}すれば、\verb+{+\verb+\large+ \verb+}+となって、
   カーソルは``\verb+{+''と``\verb+}+''の間に移ります。

 \item[随時補完] このほかにも、\verb+\bigskip+のような長いコマンドを入力す
   るときは、最初の数文字と\keytop{Ctrl}$+$\keytop{C}
   \keytop{SPC}で補ってくれます。自分の希望するものでなければ
   \keytop{Tab}を入力すれば一覧が出ます。

 \item[ギリシア文字,数式記号補完]
	    ギリシア文字や数式記号も一発で補完してくれます。ギリシア文字
	    を打ちたかったら、数式環境の中で\keytop{:}\keytop{頭のアルファ
	      ベット}を入力すればよく、数式記号は\keytop{;}に続いて割り当
	    てられてキーを打てばいけます。どのキーが対応するかはググって
	    みましょう。
            例えば$\displaystyle \frac{\partial \alpha}{\partial\beta}$も\\
	    \verb+\frac{\partial \alpha}{\partial\beta}+\\
            なんていちいち入力せずに\\
	    \keytop{;} \keytop{f} \keytop{Enter}, \keytop{;} \keytop{6}
	    \keytop{Enter}, \keytop{:} \keytop{a} \keytop{Enter},
	    \keytop{Enter}, \keytop{;} \keytop{6} \keytop{Enter},
	    \keytop{:} \keytop{b} \keytop{Enter}\\
	    となる訳です。慣れると格段にスピードが上がります。

\end{description}

\subsubsection{おまかせ改行}
itemize,enumerate,description環境の中では、改行した
ら次の行の頭には\verb+\item+とか\verb+\item[*]+とか入力することが多いわけで
す。こんなときは\keytop{Esc}$+$\keytop{Enter}\footnote{これでうまくいかない場合は、\keytop{Alt}$+$\keytop{Enter}}で勝手に\verb+\item+を補って
くれます。

\subsubsection{プロセス起動}
{\LaTeX}のソースを編集中、\keytop{Ctrl}$+$\keytop{C} \keytop{Ctrl}$+$\keytop{T}と入力すると、ミニバッファに
\begin{screen}
\begin{verbatim}
J)latex R)egion E)nv B)ibtex mk(I)dx latex+p(D)f K)ill P)review V)iewErr L)pr
\end{verbatim}
\end{screen}
のように表示されます。この説明に従って、例えば\keytop{J}と入力すれば{\LaTeX}のタイプセットが始まりますし、\keytop{P}とすればxdviが起動します。極めつけに\keytop{D}と入力すると、タイプセットした上でpdfまで作ってくれます。

\subsection{WindowsやMax OS XでEmacs$+$YaTeX}
Windowsな方のために、GNU Emacsによく似たxyzzyというテキストエディタがあ
ります。これにKaTeX(
$\leftarrow$かてふ)というメジャーモードを重ねると、Emacs$+$YaTeX
とほぼ同等の扱いができるようになります。詳しくは
\begin{screen}
 http://www.jsdlab.co.jp/\~kamei/\\
 http://osuneko.at.infoseek.co.jp/xyzzy/xyzzy.html
\end{screen}
あたりからどうぞ。

Mac OS XはそもそもUNIX系OSなので、Emacsは比較的簡単にインストールできます。もちろんCygwinやMSYS2などのUNIX仮想環境を入れれば、Windows上でも本家GNU Emacsを動かすことができます。

\subsection{{\LaTeX}の統合執筆環境}
Emacs$+$YaTeXは、必要十分な機能が揃っていて、何よりもEmacsの操作に長けた人には最適な執筆支援環境です。一方で{\LaTeX}に特化したGUIの統合開発(執筆)環境\footnote{統合開発環境とは、プログラミングに必要なコンパイラやテキストエディタ、プレビューアをGUI(グラフィカルユーザインターフェース)でまとめたソフトウェア環境のことです。例えば、XcodeやVisual Studio,Eclipseなどが有名です。}もたくさん存在し、かなり普及しています。気になる人は調べてみると良いでしょう。統合執筆環境とEmacs$+$YaTeXの大きな違いは、マウス操作でコマンドを選択できる点と、見た目が現代的なことです。ソフトごとに特色があり好みが分かれますが、最近はTeXStudioやLyxなどが流行っているようです。Lyxに至っては、Wordのようにリアルタイムに数式や文章構造の視覚的フィードバックを得ながら{\LaTeX}文書を書くことができます。
