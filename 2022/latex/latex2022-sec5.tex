%section 5
%%%%%%%%%%%%%%%%%%%%%%%%%%%%%%%%%%%%%%%
%%%%%%%%%%%%%%%%%%%%%%%%%%%%%%%%%%%%%%%

\section{環境}
これまで見てきたコマンドたちはみな、
\begin{screen}
\begin{verbatim}
\命令
\命令{引数}
\end{verbatim}
\end{screen}
という形をしていました。例えば、\verb+\maketitle+と書けば表題が表示されま
したし、\verb+\textbf{もじもじくん}+と書けば\textbf{もじもじくん}のよう
にゴシックになったわけです。これに対して、
\begin{itemize}
 \item[-] ある領域を全部中央揃えにしたい
 \item[-] ある領域を全部引用用に段落を下げたい
\end{itemize}
ということもあります。こんなときは、
\begin{screen}
\begin{verbatim}
\begin{環境}
  ある領域
\end{環境}
\end{verbatim}
\end{screen}
という\emph{環境型のコマンド}を使います。いくつか具体例を見ていきましょう。

\subsection{左揃え・中央揃え・右揃え}
とりあえず何かお知らせみたいなものを作成したいとき、この3つはよく使うと
思います。それぞれ表\ref{tab:flush}のような環境が用意されています。
\begin{table}[htbp]
\begin{center}
\caption{行を揃える環境}
\label{tab:flush}
\begin{tabular}{ll}
\hline
種類     & 環境             \\
\hline
左揃え   & \verb+flushleft+  \\
中央揃え & \verb+center+     \\
右揃え   & \verb+flushright+ \\
\hline
\end{tabular}
\end{center}
\end{table}

これらを使って、\\
\begin{minipage}[c]{.50\textwidth}
\begin{screen}
 \small
\begin{verbatim}
\begin{flushleft}
佐藤様 \\
鈴木様
\end{flushleft}

\begin{flushright}
2022年4月1日\\
地物太郎
\end{flushright}

\begin{center}
本年度の夏期賞与について
\end{center}
\end{verbatim}
\end{screen}
\end{minipage}%
%\manerrarrow\hfill{}
$\Rightarrow$
\begin{minipage}{.45\textwidth}
\begin{shadebox}
\begin{flushleft}
佐藤様 \\
鈴木様
\end{flushleft}

\begin{flushright}
2022年4月1日\\
地物太郎
\end{flushright}

\begin{center}
本年度の夏期賞与について
\end{center}

\end{shadebox}
\end{minipage}
\vspace*{1mm}\\
などと書くことができます。

\subsection{箇条書き}
長い長い文書の中に箇条書きを放り込むと、メリハリをつけることができます。
{\LaTeX}には
\begin{description}
 \item[itemize環境] 項目の頭に$\bullet$のようなマークをつける
	    \emph{記号つき箇条書き}です。入れ子にすると、それに伴って付
	    くマークも変わっていきます。
 \item[enumerate環境] 項目の頭に$1,2,3,\cdots$のような通し番号がつ
	    く\emph{番号つき箇条書き}です。入れ子にすると、それに伴って番
	    号がアルファベットになったりローマ数字になったりします。
 \item[description環境] 項目の頭が単語になっている\emph{説明つき箇
	    条書き}です。今のこの箇条書きが、まさにdescription環境
	    です。
\end{description}
という3つの箇条書き環境が用意されています。
これらはみな、
\begin{screen}
\begin{verbatim}
\begin{箇条書き環境名}
  \item[オプション]
\end{箇条書き環境名}
\end{verbatim}
\end{screen}
のように使います。

例えば、itemize環境を使うと、\\
\begin{minipage}[c]{.50\textwidth}
\begin{screen}
\small
\begin{verbatim}
我々の太陽系に属する天体は
\begin{itemize}
 \item planet
 \item dwarf planet
  \begin{itemize}
   \item ex) 冥王星,エリス
  \end{itemize}
 \item small solar system bodies
  \begin{itemize}
   \item 小惑星
    \begin{itemize}
     \item ex) リュウグウ
    \end{itemize}
   \item 彗星
   \item ほとんどのTNO
   \item その他の小天体
  \end{itemize}
\end{itemize}
に分類される(IAU Resolution 5A)
\end{verbatim}
\end{screen}
\end{minipage}%
%\manerrarrow\hfill{}
$\Rightarrow$
\begin{minipage}{.45\textwidth}
\begin{shadebox}
我々の太陽系に属する天体は
\begin{itemize}
 \item planet
 \item dwarf planet
  \begin{itemize}
   \item ex) 冥王星,エリス
  \end{itemize}
 \item small solar system bodies
  \begin{itemize}
   \item 小惑星
    \begin{itemize}
     \item ex) リュウグウ
    \end{itemize}
   \item 彗星
   \item ほとんどのTNO
   \item その他の小天体
  \end{itemize}
\end{itemize}
に分類される(IAU Resolution 5A)
\end{shadebox}
\end{minipage}
\vspace*{1mm}\\
のような箇条書きが作れます。入れ子に入るにつれて記号が変化している様子が
わかると思います。また、description環境は、次のように\verb+\item+
の後のオプションの部分に項目名を書いてやります。\\

\begin{minipage}[c]{.50\textwidth}
\begin{screen}
\small
\begin{verbatim}
地球惑星物理学演習では以下の内容を修得します:
\begin{description}
 \item[UNIX] 計算機の前で途方にくれないようにします
 \item[Fortran] Fortran90のプログラミングを学びます
 \item[Python] Pythonのプログラミングを学びます
 \item[行列] 逆行列の計算・連立1次方程式の計算方法を学びます
 \item[時間発展] 時間発展問題のシミュレーションをします
 \item[データ解析] 時系列データの解析の基礎を学びます
\end{description}
\end{verbatim}
\end{screen}
\end{minipage}%
%\manerrarrow\hfill{}
$\Rightarrow$
\begin{minipage}{.50\textwidth}
\begin{shadebox}
地球惑星物理学演習では以下の内容を修得
します:
\begin{description}
 \item[UNIX] 計算機の前で途方にくれない
           ようにします
 \item[Fortran] Fortran90のプログラミ
           ングを学びます
 \item[Python] Pythonのプログラミングを学びます
 \item[行列] 逆行列の計算・連立1次方程
           式の計算方法を学びます
 \item[時間発展] 時間発展問題のシミュレー
           ションをします
 \item[データ解析] 時系列データの解析の基
           礎を学びます
\end{description}
\end{shadebox}
\end{minipage}\\
\vspace*{5mm}

\subsection{引用}
先ほど、単語や文の引用の仕方を説明しました。しかし文といっても段落丸ご
ととなると話は変わってきます。例えば、\\
\begin{minipage}[c]{.50\textwidth}
\begin{screen}
\small
\begin{verbatim}
\emph{Aki and Richards} [2002]には,``A faulting
 source is an event  associated with an internal
 surface, such as slip across a fracture  plane.
 A volume source is an event associated with an
 internal volume, such as a sudden (explosive)
expansion throughout a volumetric source
 region. We shall find that a unified treatment
 of both source type is possible,the common link
 being the concept of an internal surface  across
 which discontinuities can occur in displacemant
 (for the  faulting source) or in strain (for the
 volume source).''というくだりがある。
\end{verbatim}
\end{screen}
\end{minipage}%
%\manerrarrow\hfill{}
$\Rightarrow$
\begin{minipage}{.45\textwidth}
\begin{shadebox}
\emph{Aki and Richards} [2002]には,``A faulting source is an event
 associated with an internal surface, such as slip across a fracture
 plane. A volume source is an event associated with an internal volume,
 such as a sudden (explosive) expansion throughout a volumetric source
 region. We shall find that a unified treatment of both source type is
 possible,the common link being the concept of an internal surface
 across which discontinuities can occur in displacemant (for the
 faulting source) or in strain (for the volume source).''というくだりがあ
 る。
\end{shadebox}
\end{minipage}
\vspace*{1mm}\\
なんてのはどう見てもちょっと勘弁して欲しいです。段落丸ごとを引用するとき
は、やはり別のかたまりになっていたほうが見やすいと思います。

そこで、段落の引用用に行頭の字下
げをしないquoteと字下げをするquotationという環境が用意され
ています。複数段落を引用するのなら、quotationのほうがよいでしょう。
これを使うと上の例では\\
\begin{minipage}[c]{.50\textwidth}
\begin{screen}
\small
\begin{verbatim}
\emph{Aki and Richards} [2002]には,
\begin{quote}
 A faulting source is an event  associated with
 an internal surface, such as slip across a
 fracture plane. A volume source is an event
 associated (中略) surface across which
 discontinuities can occur in displacemant
 (for the  faulting source) or in strain (for the
 volume source).
\end{quote}
というくだりがある。
\end{verbatim}
\end{screen}
\end{minipage}%
%\manerrarrow\hfill{}
$\Rightarrow$
\begin{minipage}{.45\textwidth}
\begin{shadebox}
\emph{Aki and Richards} [2002]には,
\begin{quote}
 A faulting source is an event  associated with
 an internal surface, such as slip across a
 fracture plane. A volume source is an event
 associated (中略) surface across which
 discontinuities can occur in displacemant
 (for the  faulting source) or in strain (for the
 volume source).
\end{quote}
というくだりがある。
\end{shadebox}
\end{minipage}
\vspace*{1mm}\\
となります。

\subsection{逐語引用}
これからレポート等で文書にプログラミングのソースコードを
載せる機会があるかもしれません。そうした時は、単にソースコード
をコピペで貼り付けても{\LaTeX}はそれをソースコードとして認識せず
空白や改行を勝手につぶしてしまい、全くコードとしては読めなくなってしまい
ます。これはquote環境を使っても解決できません。\\
\begin{minipage}[c]{.45\textwidth}
\begin{screen}
 \small
\begin{verbatim}
以下にFortranプログラムのサンプルを載せる。
program main
  implicit none
  integer ::i

  do i=1,100
     write (*,*) 'Hello, World !'
  end do

  stop
end program main
以上。
\end{verbatim}
\end{screen}
\end{minipage}%
%\manerrarrow\hfill{}
$\Rightarrow$
\begin{minipage}{.50\textwidth}
\begin{shadebox}
以下にFortranプログラムのサンプルを載せる。
program main
  implicit none
  integer ::i

  do i=1,100
     write (*,*) 'Hello, World !'
  end do

  stop
end program main
以上。
\end{shadebox}
\end{minipage}
\vspace*{1mm}\\
だからといって読めるように改行等の装飾を施して
いたのでは日が暮れてしまいます。
こういう時はverbatim環境を用いましょう。
verbatim環境では、内容が改行や空白も
{\LaTeX}処理されずに
``そのまま''に出力されます。\\
\begin{minipage}[c]{.45\textwidth}
\begin{screen}
\small
\begin{alltt}
以下にFortranプログラムのサンプルを載せる。
\verb+\begin{verbatim}+
program main
  implicit none
  integer ::i

  do i=1,100
     write (*,*) 'Hello, World !'
  end do

  stop
end program main
\verb+\end{verbatim}+
以上.
\end{alltt}
\end{screen}
\end{minipage}
%\manerrarrow\hfill{}
$\Rightarrow$
\begin{minipage}{.50\textwidth}
\begin{shadebox}
以下にFortranプログラムのサンプルを載せる。
\begin{verbatim}
program main
  implicit none
  integer ::i

  do i=1,100
     write (*,*) 'Hello, World !'
  end do

  stop
end program main
\end{verbatim}
以上。
\end{shadebox}
\end{minipage}
\vspace*{1mm}\\
似たような働きを持つものに\verb+\verb+コマンドがあります。
これは改行を含まない内容を``+''で囲んでそのまま出力します。
このverbatim環境は囲んだ部分を{\LaTeX}処理しないので、
texのソースを内部に書いても``そのまま''出力します。
このテキストももちろんtexで作られていますが、作成時
にこのverbatim環境は至る所で使われています。どの部分
に使われているかは言わずもがなですよね...。


\subsection{練習}
プロジェクトtex\_exerciseの中にある\underline{exercise2.tex}と
\underline{/home2/takata2022/exercise/}にある
\underline{exercise2a.pdf}
を見比べてみてください。
どうやらpdfファイルと同じ文章を作りたかったようですが、
急いで作ったのか見栄えがいまいちです。ここまでで学習したコマンドを用いて
この文書をきれいに修正してみてください。またitemizeをenumerateに
してみるのもいいでしょう。

\pagebreak
