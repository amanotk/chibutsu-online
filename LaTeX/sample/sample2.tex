% LuaLaTeXでの利用の場合を想定
\documentclass[lualatex]{jlreq}
% プリアンブルで著者名等を設定すると管理しやすい
\title{タイトル}
\author{著者名}
\date{日付}
% コンパイルした日の日にちを出したい場合には
% \dateを使わない
% 日にちを出したくない場合には
% \date{}
% とする

% パッケージの利用
\usepackage{scsnowman}

\begin{document}
	% タイトル(タイトル・著者名・日付)の出力
	\maketitle

	% 目次の出力
	% 一回では反映されない点に注意
	\tableofcontents
	
	% 節見出し
	\section{節見出し}
	適当なイントロ
	\subsection{小節見出し1}
	ほげほげ
	\subsection{小節見出し2}
	ふがふが

	% 書体の変更
	このようにして\textgt{ゴシック体}にします.\\
	違う属性なら\textbf{\textsf{hoge}}みたいにもできます.

	% 文字サイズ変更
	こんなふうに文字サイズを{\LARGE 大きく}したり{\footnotesize 小さく}できます

	% 箇条書き
	% itemize環境の場合
	\begin{itemize}
		\item 要素1
		\item 要素2
	\end{itemize}
	%description環境の場合
	\begin{description}
		\item[大気海洋分野] 適当な説明
		\item[固体地球分野] 適当な説明
		\item[宇宙惑星分野] 適当な説明
	\end{description}
	\newpage
	%入れ子にすると頭の記号が変わる
	\begin{itemize}
		\item 要素1
		\begin{itemize}
			\item 要素1-1
			\item 要素1-2
		\end{itemize}
		\item 要素2
	\end{itemize}

	% 脚注
	適当な文章に脚注をつけてみましょう.
	\footnote{このように脚注をつけられます}

	% 引用
	% 簡単な引用
	ほげ山ほげ尾の「ほげの科学」には
	\begin{quotation}
		適当な文章
	\end{quotation}
	というくだりがある.

	% 参考文献
	% 本文
	ほげ山ほげ尾の「ほげの科学」~\cite{ほげ},ふが川ふが実の「ふが学」~\cite{ふが}によると...

	% パッケージの利用例
	\scsnowman

	% 参考文献
	\begin{thebibliography}{9}
		\bibitem{ほげ} ほげ山ほげ尾『ほげの科学』ほげ出版
		\bibitem{ふが} ふが川ふが実『ふが学』ふが出版
	\end{thebibliography}
\end{document}