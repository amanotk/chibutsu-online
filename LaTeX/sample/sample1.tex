%% (u)pLaTeXの場合は下のコメントアウトは外す
% \RequirePackage{plautopatch}

% --- ドキュメントクラスの指定 ---
%% pLaTeXの場合
% \documentclass[platex,dvipdfmx]{jlreq}
%% upLaTeXの場合
% \documentclass[uplatex,dvipdfmx]{jlreq}
%%LuaLaTeXの場合
\documentclass[lualatex]{jlreq}
% --- ここからプリアンブル ---

%% (u)pLaTeXの場合は以下も入れる
% \usepackage[T1]{fontenc}
% \usepackage{lmodern}

%%%%%%%%%%%
\NewDocumentCommand\somecs{O{t} m}{%
	$\frac{\partial #2}{\partial #1}$%
}

% --- ここまでプリアンブル ---
\begin{document}
% --- ここから本文 ---
	% 命令の練習
	% 命令の基本
	\LaTeX の練習\\  %コマンドの後に半角スペース->正しく動く
	{\LaTeX}の練習\\ %コマンドを{}で囲う->正しく動く
	\LaTeX{}の練習\\ %コマンドの後に{}を入力->正しく動く
	%\LaTeXの練習   %コマンドに続けて地の文を入力->エラー

	% 引数ありの命令
	\somecs{f}
	\somecs[x]{f}

	% 改行の練習
	このように入力ファイルで改行しても
	出力では改行されません.

	空白行を入れることで改行(改段落)されます.

	% 空白の練習
	全角の空白なら     しっかり出力される\\
	半角空白だと     1個しか出力されない\\
	これなら\ \ \ \ \ しっかり出る\\
	これでも~~~~~しっかり出力される

	% 特殊な記号の例
	ブレース(左):\{ \\
	バックスラッシュ:\textbackslash \\
	(u)pLaTeXの場合,fonstencとlmodernを入れ忘れると,
	半角のアングルブラケット<>など一部の記号は正しく出力されない.

% --- ここまで本文 ---
\end{document}
% --- ここから下には何も書かない ---