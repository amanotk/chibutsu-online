\documentclass{jsarticle}
\usepackage{ascmac}
\usepackage{fancybox}
\usepackage[top=4cm, bottom=4cm, left=4.7cm, right=4.7cm]{geometry}
\usepackage{amsmath}
\def \pa{\partial}
\begin{document}
\begin{center}
\ovalbox{課題名、名前、学籍番号、提出日を書いて下さい。}
\end{center}
\vspace*{1cm}

宇宙空間はプラズマと呼ばれる電離気体で満たされている。
以下で示す電磁流体の方程式を用いることで、我々は宇宙空間での現象をしばしば説明することができる
\footnote{これらの式を用いるだけで、太陽コロナや降着円盤、その他の宇宙現象をシミュレーションなどで再現することが可能である。宇宙空間は夢と希望とプラズマに満ちている。}。
\begin{gather}
  \frac{\pa \rho}{\pa t} + \nabla \cdot (\rho \vec{v})=0\\
  \frac{\pa}{\pa t}(\rho \vec{v})+\vec{v}\cdot\nabla(\rho \vec{v})=-\nabla p + \frac{1}{c}\vec{j} \times \vec{B}\\
  \frac{\pa}{\pa t}\left( \frac{p}{\gamma - 1} \right) + \nabla \cdot \left( \frac{\gamma}{\gamma -1}p\vec{v} \right) = (\vec{v}\cdot \nabla)p + \frac{4\pi}{c^2}\bar{\eta}\vec{j}^2\\
  \frac{\pa \vec{B}}{\pa t} = -c\nabla \times \vec{E}\\
  \vec{j} = \frac{c}{4\pi}\nabla \times \vec{B}\\
  \vec{E}+\frac{\vec{v}\times \vec{B}}{c}=\frac{4\pi}{c^2}\bar{\eta}\vec{j}^2
\end{gather}
以下の表に各式とその式の示す物理的意味をまとめておこう。
\begin{center}
\begin{tabular}{|c|c|}
  \hline
  式番号 & 物理的意味\\
  \hline
  (1) & 質量保存\\
  \hline
  (2) & 運動量保存(運動方程式)\\
  \hline
  (3) & エネルギー保存(熱流方程式)\\
  \hline
  (4) & Faradayの電磁誘導の法則\\
  \hline
  (5) & Ampereの法則\\
  \hline
  (6) & Ohmの法則\\
  \hline
\end{tabular}

\vspace*{2.5cm}

\ovalbox{ここに図を貼り付ける}
\end{center}


\end{document}

