%section 8
%%%%%%%%%%%%%%%%%%%%%%%%%%%%%%%%%%%%%%%
%%%%%%%%%%%%%%%%%%%%%%%%%%%%%%%%%%%%%%%

\section{コマンドの編集}
ここまで読んでくれば、{\TeX}を使って一通りの作業ができ、レポートなどの文書
も書けるようになると思います。しかし記号の一つ一つに対してコマンドが割り
振られている、という{\TeX}の特性上どうしても一つの文書を書き上げるまで多
くの文字を打ち込まなければなりません。ただ面倒臭いと言うだけならまだしも、
レポートなどになると偏微分の記号
$\displaystyle \frac{\partial}{\partial t}$
などがしょっちゅうでてきて、これをいちいち
\verb+\frac{\partial}{\partial t}+などと打ち込んでいるのは辟易してきます。
このような場合には自分で偏微分の記号を一発で表示させるコマンドを作ってしまう方が便利です。
このような場合に非常に重宝するのが\verb+\def+コマンドです。
たぶん\verb+define+かなんかを短縮したものでしょう。
さて、プリアンブルに例えば、
\verb+\def\deln#1{\frac{\partial}{\partial {\mbox{$#1$}}}}+
と打ってみましょう。
これはちょっと長めですが、このあとの幸せを考えるとまあ耐えられます。
さてこの一行の意味ですが、\verb+\def+というコマンドで直後の\verb+\deln+と
いうコマンドを定義せよ、と命令しているわけです。\verb+\deln+の直後の
\verb+#1+は引数を入れる欄を作っています。わざわざ真似て新作のコマンドの名前を
\verb+\deln+にする必要はないですが、
ここでは偏微分の``デル''の分子に``微分される関数が何もつきませんよ(nothing)''
のつもりでこの名前にしてます。別に
\verb+\def\riderkick#1+云々でも構いません。こうやった上で微分したい変数を
引数にいれて\\
\begin{minipage}[c]{.50\textwidth}
\begin{screen}
\small
\begin{verbatim}
この様に\\
$\deln{x}$\\
とすれば
\end{verbatim}
\end{screen}
\end{minipage}%
%\manerrarrow\hfill{}
$\Rightarrow$
\begin{minipage}{.45\textwidth}
\begin{shadebox}
この様に\\
$\frac{\partial}{\partial x}$\\
とすれば
\end{shadebox}
\end{minipage}\\
\vspace*{5mm}
ちゃんと偏微分が表示されています。これはほんの一例ですので色々と各自試し
て見てください。微分の記号くらいは充実させておくと便利です。なお
\verb+\def+の代わりに
\verb+\newcommand+でも似たような操作ができるみたいです。興味がある人は調べてみてください。

\pagebreak